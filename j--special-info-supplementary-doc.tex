Except as specified below, special information and supplementary
documentation must be included as part of the Project Description (or
part of the budget justification), if it is relevant to determining
the quality of the proposed work. Information submitted in the
following areas is not considered part of the 15-page Project
Description limitation. This Special Information and Supplementary
Documentation section also is not considered an appendix. Specific
guidance on the need for additional documentation may be obtained from
the organization's sponsored projects office or in the references
cited below.


\section*{Postdoctoral Mentoring Plan}
Postdoctoral Researcher Mentoring Plan. Each proposal29 that requests
funding to support postdoctoral researchers30 must include, as a
supplementary document, a description of the mentoring activities that
will be provided for such individuals. If a Postdoctoral Researcher
Mentoring Plan is required, FastLane will not permit submission of a
proposal if the Plan is missing. In no more than one page, the
mentoring plan must describe the mentoring that will be provided to
all postdoctoral researchers supported by the project, irrespective of
whether they reside at the submitting organization, any subrecipient
organization, or at any organization participating in a simultaneously
submitted collaborative project. Proposers are advised that the
mentoring plan must not be used to circumvent the 15-page Project
Description limitation. See GPG Chapter II.D.5 for additional
information on collaborative proposals. Mentoring activities provided
to postdoctoral researchers supported on the project will be evaluated
under the Broader Impacts review criterion.

Examples of mentoring activities include, but are not limited to:
career counseling; training in preparation of grant proposals,
publications and presentations; guidance on ways to improve teaching
and mentoring skills; guidance on how to effectively collaborate with
researchers from diverse backgrounds and disciplinary areas; and
training in responsible professional practices.


\section*{Data Management Plan}
Plans for data management and sharing of the products of
research. Proposals must include a supplementary document of no more
than two pages labeled "Data Management Plan". This supplementary
document should describe how the proposal will conform to NSF policy
on the dissemination and sharing of research results (see AAG Chapter
VI.D.4), and may include:

\begin{enumerate}
\item the types of data, samples, physical collections, software,
  curriculum materials, and other materials to be produced in the
  course of the project;

\item the standards to be used for data and metadata format and
  content (where existing standards are absent or deemed inadequate,
  this should be documented along with any proposed solutions or
  remedies);

\item policies for access and sharing including provisions for
  appropriate protection of privacy, confidentiality, security,
  intellectual property, or other rights or requirements;

\item policies and provisions for re-use, re-distribution, and the
  production of derivatives; and

\item plans for archiving data, samples, and other research products,
  and for preservation of access to them.
\end{enumerate}

Data management requirements and plans specific to the Directorate,
Office, Division, Program, or other NSF unit, relevant to a proposal
are available at: \url{http://www.nsf.gov/bfa/dias/policy/dmp.jsp}. If
guidance specific to the program is not available, then the
requirements established in this section apply.

Simultaneously submitted collaborative proposals and proposals that
include subawards are a single unified project and should include only
one supplemental combined Data Management Plan, regardless of the
number of non-lead collaborative proposals or subawards
included. FastLane will not permit submission of a proposal that is
missing a Data Management Plan.

A valid Data Management Plan may include only the statement that no
detailed plan is needed, as long as the statement is accompanied by a
clear justification. Proposers who feel that the plan cannot fit
within the limit of two pages may use part of the 15-page Project
Description for additional data management information. Proposers are
advised that the Data Management Plan must not be used to circumvent
the 15-page Project Description limitation. The Data Management Plan
will be reviewed as an integral part of the proposal, considered under
Intellectual Merit or Broader Impacts or both, as appropriate for the
scientific community of relevance.

