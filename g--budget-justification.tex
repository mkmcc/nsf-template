Each proposal must contain a budget for each year of support
requested. The budget justification must be no more than three pages
per proposal. The amounts for each budget line item requested must be
documented and justified in the budget justification as specified
below. For proposals that contain a subaward(s), each subaward must
include a separate budget justification of no more than three pages.

The proposal may request funds under any of the categories listed so
long as the item and amount are considered necessary, reasonable,
allocable, and allowable under 2 CFR \S 200, Subpart E, NSF policy,
and/or the program solicitation. For-profit entities are subject to
the cost principles contained in the Federal Acquisition Regulation,
Part 31. Amounts and expenses budgeted also must be consistent with
the proposing organization's policies and procedures and cost
accounting practices used in accumulating and reporting costs.

\subsection*{A. Senior personnel}
\begin{itemize}
\item[A1.] salary support for the PI

\item[A2.] salary support for other senior personnel
\end{itemize}


\subsection*{B. Other personnel}
\begin{itemize}
\item[B1.] postdocs

\item[B2.] other professionals

\item[B3.] graduate students

\item[B4.] undergraduate students
\end{itemize}

\subsection*{C. Fringe Benefits}
Fringe benefits are calculated at a rate X\% for faculty, Y\% for
postdocs, and Z\% for grad students.

\subsection*{D. Equipment}
Equipment is defined as tangible personal property (including
information technology systems) having a useful life of more than one
year and a per-unit acquisition cost which equals or exceeds the
lesser of the capitalization level established by the proposer for
financial statement purposes, or \$5,000. It is important to note that
the acquisition cost of equipment includes modifications, attachments,
and accessories necessary to make the property usable for the purpose
for which it was purchased. Items of needed equipment must be
adequately justified, listed individually by description and estimated
cost.

Allowable items ordinarily will be limited to research equipment and
apparatus not already available for the conduct of the work. General
purpose equipment such as office equipment and furnishings, and
information technology equipment and systems are typically not
eligible for direct cost support. Special purpose or scientific use
computers or associated hardware and software, however, may be
requested as items of equipment when necessary to accomplish the
project objectives and not otherwise reasonably available. Any request
to support such items must be clearly disclosed in the proposal
budget, justified in the budget justification, and be included in the
NSF award budget.

\subsection*{E. Travel}
Travel and its relation to the proposed activities must be specified,
itemized and justified by destination and cost. Funds may be requested
for field work, attendance at meetings and conferences, and other
travel associated with the proposed work, including subsistence. In
order to qualify for support, however, attendance at meetings or
conferences must be necessary to accomplish proposal objectives, or
disseminate its results. Travel support for dependents of key project
personnel may be requested only when the travel is for a duration of
six months or more either by inclusion in the approved budget or with
the prior written approval of the cognizant NSF Grants
Officer. Temporary dependent care costs above and beyond regular
dependent care that directly result from travel to conferences are
allowable costs provided that the conditions established in 2 CFR \S
200.474 are met.

Allowance for air travel normally will not exceed the cost of
round-trip, economy airfares. Persons traveling under NSF grants must
travel by US-Flag Air carriers, if available.



\subsection*{G. Other Direct Costs}
Any costs proposed to an NSF grant must be allowable, reasonable and
directly allocable to the supported activity. The budget must identify
and itemize other anticipated direct costs not included under the
headings above, including materials and supplies, publication costs,
computer services and consultant services. Examples include aircraft
rental, space rental at research establishments away from the grantee
organization, minor building alterations, payments to human subjects,
and service charges. Reference books and periodicals may be charged to
the grant only if they are specifically allocable to the project being
supported by NSF.

\subsection*{I. Indirect Costs}
University Overhead
